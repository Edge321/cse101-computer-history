%%%%%%%%%%%%%%%%%%%%%%%%%%%%%%%%%%%%%%%%%%%%%%%%%%%%%%%%%%%%%%%%%%%%%%%%%%%%%%%%

% IEEEconf.cls file must exist in the same directory as the TeX file you want to compile
\documentclass[letterpaper, 10 pt, conference]{IEEEconf}

\title{\LARGE \bf
COMPUTER HISTORY\\
\large Computers Makes Movies: Shrek
}

\author{Group 4\\
\small Edgar Pallares\\
\small Angel Ruiz\\
\small Korbin Shelley\\
}

% Image/graphics support
\usepackage{graphicx}
\graphicspath{ {./images/} }

% Formatting for lists
\usepackage{enumitem}
\usepackage{hyperref}

% Formatting for media
\usepackage{float}
\restylefloat{table}
\restylefloat{figure}

\begin{document}


\maketitle
\thispagestyle{empty}
\pagestyle{empty}


%%%%%%%%%%%%%%%%%%%%%%%%%%%%%%%%%%%%%%%%%%%%%%%%%%%%%%%%%%%%%%%%%%%%%%%%%%%%%%%%
\section*{ABSTRACT}
\textit{
This document is a basic template for writing conference-style
reports in LaTeX. You will use this template when writing
your report; you will need to replace all text (excluding
section headers or preamble information) with the content
of your report.
}

%%%%%%%%%%%%%%%%%%%%%%%%%%%%%%%%%%%%%%%%%%%%%%%%%%%%%%%%%%%%%%%%%%%%%%%%%%%%%%%%
\section{INTRODUCTION}

This document is an example of Assignment 3 in CSE/IT101.
You should describe what topic you chose and
why you chose that topic. Also provide a summary of the
topic, in a general sense. This section should describe any
background information that the reader needs to know to
understand your topic.

%%%%%%%%%%%%%%%%%%%%%%%%%%%%%%%%%%%%%%%%%%%%%%%%%%%%%%%%%%%%%%%%%%%%%%%%%%%%%%%%
\section{TIME PERIOD}

You should describe the time period in which your topic was
invented or used here. Also include the context for why your
topic was created or for how it is used. Any specific historical
information should be included here.

%%%%%%%%%%%%%%%%%%%%%%%%%%%%%%%%%%%%%%%%%%%%%%%%%%%%%%%%%%%%%%%%%%%%%%%%%%%%%%%%
\section{COMPUTER HARDWARE}

Since Shrek was released in 2001, it's safe to say computers that
are considered "ancient" nowadays would have been used for this
movie's production. Even though the movie was released in 2001, production
for computer animation of the movie started in 1996. DreamWorks Animation 
is partners with HP, and used all technology related to HP to make Shrek 
happen. Being partners with HP, it is assumed they used HP Pavilion PCs at
the time of creating the animation of Shrek. The PCs they used looked like
Figure \ref{fig:PC}, and contained the following PC parts in Table \ref{tbl:Parts}.

\begin{figure}[h!]
\centering
\includegraphics[width=0.4\textwidth]{HP}
\caption{HP Pavilion PC, 1995}
\label{fig:PC}
\end{figure} 

\begin{table}[h!]
\begin{center}
\begin{tabular}{||c | c ||} 
\hline
  & PC Parts\\ [0.5ex]
\hline\hline
CD Drive & Quad-Speed CD-ROM \\ 
\hline
Speakers & Altec Lansing \\
\hline
OS & Windows 95 \\
\hline
CPU & Pentium 75 MHz Processor\\
\hline
RAM & 8 MB\\
\hline
Hard Drive & 850 MB\\
\hline
\end{tabular}
\caption{Parts for HP Pavilion}
\label{tbl:Parts}
\end{center}
\end{table}

\section{COMPUTER SOFTWARE}

Describe the software used for your chosen topic if any,
and state any uses of the software that your topic had.
If your topic does not have or use software, describe why it
doesn't use software and how it functions without it.

\section{CONCLUSION}

Conclude your research paper with any reflections on what you
learned about your topic. Was this what you expected to find?
Did you find any facts that surprised you? You may add other
personal reflections about the topic here.

\section*{REFERENCES}

\begin{enumerate}[label={[\arabic*]}]
\item \url{https://dreamworks.fandom.com/wiki/Shrek_(film)}
\item \url{https://www.youtube.com/watch?v=yDyXxQ2EWjY}
\item \url{http://www.hp.com/hpinfo/abouthp/histnfacts/museum/personalsystems/0034/0034threeqtr.html}
\item \url{https://www8.hp.com/us/en/hp-information/about-hp/history/hp-timeline/timeline.html}
\end{enumerate}

\end{document}

