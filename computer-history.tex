%%%%%%%%%%%%%%%%%%%%%%%%%%%%%%%%%%%%%%%%%%%%%%%%%%%%%%%%%%%%%%%%%%%%%%%%%%%%%%%%

% IEEEconf.cls file must exist in the same directory as the TeX file you want to compile
\documentclass[letterpaper, 10 pt, conference]{IEEEconf}

\title{\LARGE \bf
COMPUTER HISTORY\\
\large Computers Makes Movies: Shrek
}

\author{Group 4\\
\small Edgar Pallares\\
\small Angel Ruiz\\
\small Korbin Shelley\\
}

% Image/graphics support
\usepackage{graphicx}
\graphicspath{ {./images/} }

% Formatting for lists
\usepackage{enumitem}
\usepackage{hyperref}

% Formatting for media
\usepackage{float}
\restylefloat{table}
\restylefloat{figure}

\begin{document}


\maketitle
\thispagestyle{empty}
\pagestyle{empty}


%%%%%%%%%%%%%%%%%%%%%%%%%%%%%%%%%%%%%%%%%%%%%%%%%%%%%%%%%%%%%%%%%%%%%%%%%%%%%%%%


%%%%%%%%%%%%%%%%%%%%%%%%%%%%%%%%%%%%%%%%%%%%%%%%%%%%%%%%%%%%%%%%%%%%%%%%%%%%%%%%
\section{INTRODUCTION}

Computers have brought new, ground breaking techniques to a multitude of fields creating some of the most memorable moments for many generations. One of those fields is computers making movies, where the first computer made movies became some of the biggest pop icons for several generations. One of these multi generational movies is Shrek. Shrek came out in 2001 as one of the first computer made movie working with HP and Pacific Data Images to bring the  movie to life after 5 long years of work.

%%%%%%%%%%%%%%%%%%%%%%%%%%%%%%%%%%%%%%%%%%%%%%%%%%%%%%%%%%%%%%%%%%%%%%%%%%%%%%%%
\section{TIME PERIOD}

Development of Shrek began in November 1995 after filming rights were
acquired. At the time, Shrek was initially planned to be partially
live-action and computer generated. At this time, Toy Story was just
released. Toy Story was not only the first computer generated film, but
also one of the most successful movies to air, boasting a 100\% approval
rating on Rotten Tomatoes and is often considered one of the best films
ever made. After compiling Shrek, Dreamworks determined the animation style
was insuitable and did not work. They then partnered with the crew behind
the film Antz which was nearing it's theatric release. Upon partnership
the film proceeded to follow the trend of being an entirely computer
generated film and on May 18, 2001 became the 9th computer generated film.
%%%%%%%%%%%%%%%%%%%%%%%%%%%%%%%%%%%%%%%%%%%%%%%%%%%%%%%%%%%%%%%%%%%%%%%%%%%%%%%%
\section{COMPUTER HARDWARE}

Since Shrek was released in 2001, it's safe to say computers that
are considered "ancient" nowadays would have been used for this
movie's production. Even though the movie was released in 2001, production
for computer animation of the movie started in 1996. DreamWorks Animation 
is partners with HP, and used all technology related to HP to make Shrek 
happen. Being partners with HP, it is assumed they used HP Pavilion PCs at
the time of creating the animation of Shrek. The PCs they used looked like
Figure \ref{fig:PC}, and contained the following PC parts in Table \ref{tbl:Parts}.

\begin{figure}[h!]
\centering
\includegraphics[width=0.4\textwidth]{HP}
\caption{HP Pavilion PC, 1995}
\label{fig:PC}
\end{figure} 

\begin{table}[h!]
\begin{center}
\begin{tabular}{||c | c ||} 
\hline
  & PC Parts\\ [0.5ex]
\hline\hline
CD Drive & Quad-Speed CD-ROM \\ 
\hline
Speakers & Altec Lansing \\
\hline
OS & Windows 95 \\
\hline
CPU & Pentium 75 MHz Processor\\
\hline
RAM & 8 MB\\
\hline
Hard Drive & 850 MB\\
\hline
\end{tabular}
\caption{Parts for HP Pavilion}
\label{tbl:Parts}
\end{center}
\end{table}

\section{COMPUTER SOFTWARE}


When Shrek was first being developed in 1996 there were not many options for computer animation both professionally and publicly. Though because Dream Works partnered with Pacific Data Images they were able to gain access to PDI's specialty software, Fluid Animation Systems. Fluid Animation Systems is a software system that is used to animate objects, characters, scenes, etc. in a realistic way. Dream Works did not solely use PDI's Fluid Animation Systems but they also utilized public animation software such as Adobe Photoshop 4.0 and Maya. Photoshop was used to draw the first character illustrations and later base images for the character. While Maya is a software system that is used to create fine and detailed animations. Mayas main purpose in the creation of Shrek was to animate the hair of Fiona and Lord Farquaad along with all of the clothing folds. With out the use of each of these softwares it would have been much more difficult to create such a detailed and fluid movie.

\section{CONCLUSION}
 
Seeing how far computers have come in making movies compared to what they where 25 years ago when Shrek was first being made is astounding. Both the software and hardware where limited to very few options and with these limitations Shrek came out to be something entirely different than intended. The hardware used for Shrek compared to now is slow and has less memory compared to a computer that would be used for current computer animations. While the software available for it was even more limited. 

 
\section*{REFERENCES}

\begin{enumerate}[label={[\arabic*]}]
\item \url{https://dreamworks.fandom.com/wiki/Shrek_(film)}
\item \url{https://www.youtube.com/watch?v=yDyXxQ2EWjY}
\item \url{http://www.hp.com/hpinfo/abouthp/histnfacts/museum/personalsystems/0034/0034threeqtr.html}
\item \url{https://www8.hp.com/us/en/hp-information/about-hp/history/hp-timeline/timeline.html}
\item \url{http://jimhillmedia.com/editor_in_chief1/b/jim_hill/archive/2004/05/17/how-quot-shrek-quot-went-from-being-a-train-wreck-to-one-for-the-record-books.aspx}
\item \url{https://www.imdb.com/title/tt0114709/}
\item \url{https://en.wikipedia.org/wiki/List_of_computer-animated_films}
\item \url{https://en.wikipedia.org/wiki/Shrek#Animation}
\item \url{https://www.webdesignmuseum.org/old-software/graphic-software/adobe-photoshop-4-0}
\item \url{https://www.digitalmediafx.com/Shrek/shrekfaq.html#:~:text=Q%3A%20What%20software%20was%20used,programs%20available%20to%20the%20public.}
\item \url{https://en.wikipedia.org/wiki/Fluid_animation}

\end{enumerate}

\end{document}

